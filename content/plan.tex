\section{执行计划}\label{plan}

尚需完成的功能有:

\begin{itemize}
    \item 像素坐标对应
    \item 像素亮度计算
    \item 像素色度计算
    \item 均匀度评估
    \item 校正矩阵计算
    \item 校正结果预览
    \item 校正结果上传
\end{itemize}

\subsection{像素坐标对应}

像素搜索结果(见\ref{search_pixel})给出的是一系列坐标,每个坐标与显示屏/模块上的点的对应关系还不明确。对像素进行校正还需要借助人工提供的信息来找出这个这个映射关系。

难点:图像一般不水平,每一行的x、y坐标区间都可能和其他行有重合的部分,无法简单通过排序找到对应关系。

时间估算:2周

\subsection{像素亮度计算}

设计亮度表达式,计算亮度值。

难点:要考虑人的视觉是非线性的,量化亮度差1\%,在高亮度和低亮度的条件下,人眼感知到亮度差有很大区别。

时间估算:3周

\subsection{像素色度计算}

设计色度表达式,计算亮度值。

难点:人眼对红绿蓝三色的敏感度不同。

时间估算:3周

\subsection{均匀度评估}

根据每个像素的亮度、色度,计算显示屏/模块的亮度均匀性以及色度均匀性。

时间估算:2周

\subsection{校正矩阵计算}

根据每个像素的亮度、色度,计算校正矩阵,使得校正后的亮度均匀性和色度均匀性达到最优。

难点:边缘处理;坏点、暗点处理;计算速度。

时间估算:4周

\subsection{校正结果预览}

将校正矩阵应用到校色样本,实时查看校正效果。

时间估算:3周

\subsection{校正结果上传}

将校正矩阵传输到显示驱动单元。

时间估算:未知,需要先了解硬件上的实现与限制。

\clearpage
